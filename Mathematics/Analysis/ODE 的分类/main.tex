%\documentclass{article}
\documentclass[UTF8]{ctexart}
% Language setting
% Replace `english' with e.g. `spanish' to change the document language


% Set page size and margins
% Replace `letterpaper' with `a4paper' for UK/EU standard size
\usepackage[a4paper,top=2cm,bottom=2cm,left=3cm,right=3cm,marginparwidth=1.75cm]{geometry}

% Useful packages
\usepackage{amsmath}
\usepackage{graphicx}
\usepackage[colorlinks=true, allcolors=blue]{hyperref}

\title{常微分方程(ODE)的分类}
\author{Bright Moon}

\begin{document}
\maketitle
\section{常微分方程(ODE)的分类框架}
\subsection{阶数与线性的辨析}
\[F(x,y^{(0)},y^{(1)},y^{(2)},...,y^{(n)})=0\]
\subsubsection{阶数——导数的阶数}
阶数取决于导数$y^{(n)}$的最高阶数$n$。上述微分方程为\textbf{$n$阶}常微分方程。
\subsubsection{线性——导数的幂次}
如果上述微分方程可以写成关于各阶导数$(y^{(i)})^{k_i}$的多项式形式:\[y^{(n)}+a_1(x)y^{(n-1)}+a_2(x)y^{(n-2)}+...+a_n(x)y^{(0)}=f(x)\]
\par 所涉及的各阶导数$(y^{(i)})^{k_i}$在方程中的幂次$k_i$不超过一次,则是\textbf{线性}常微分方程。
\par 进一步,如果$f(x)=0$,则称线性方程是\textbf{齐次}的,否则为\textbf{非齐次}的。
\par 如果各阶导数前的系数$a_i(x)$不依赖于$x$,是常数$a_i$,则称线性方程是\textbf{常系数}。

\subsection{常微分方程与常微分方程组的辨析}
\subsubsection{“常”/“偏”——函数自变量的个数}
\textbf{一个}常微分方程的未知函数是一个一元函数$f(x)$。
\par \textbf{一个}偏微分方程的未知函数是一个多元函数$f(x_1,...,x_n)$。
\subsubsection{“组”——函数的个数}
常微分方程\textbf{组}的未知函数是多个一元函数(一个一元向量函数)\[v_1(t),...,v_n(t)\] \[\text{或}\quad \vec  v(t) = (v_1(t),...,v_n(t))\]

\subsection{可以用初等积分法求解的常微分方程(ODE)的分类}
\begin{enumerate}
    \item 一阶
    \begin{enumerate}
        \item 线性大类
        \begin{enumerate}
            \item 线性齐次\[y'+P(x)y=0\]


        \end{enumerate}
        \begin{enumerate}
            \item 线性非齐次\[y'+P(x)y=Q(x)\]
            \item 伯努利方程(仍然是一阶的,但不是线性的)(可以化为线性方程)\[y'+P(x)y=Q(x)y^{\alpha}\]
        
        \end{enumerate}
        \item 伯努利方程(仍然是一阶的,但不是线性的)(可以化为线性方程)\[y'+P(x)y=Q(x)y^{\alpha}\]
        \item 变量分离大类(包含线性与非线性)
        \begin{enumerate}
            \item 纯粹变量分离型\[y'=f(x)g(y)\]
            \item 复合一次函数型\[y'=f(ax+by+c)\]
            \item 齐次函数型\[y'=f(x,y)=h(\frac{y}{x})\]
            \item 一次分式型\[y'=f(\frac{a_1x+b_1y+c_1}{a_2x+b_2y+c_2})\]

        \end{enumerate}
        \item 可以降为一阶的大类(包含线性与非线性)
        \begin{enumerate}
            \item 二阶缺项
            \begin{enumerate}
                \item 不显含$y$\[F(x,y',y'')\]
                \item 不显含$x$\[F(y,y',y'')\]

            \end{enumerate}

        \end{enumerate}
        \item 全微分方程(恰当方程)大类
        \begin{enumerate}
            \item 纯粹的全微分方程\[y'=-\frac{P(x,y)}{Q(x,y)}\]\[\text{或:}\ P(x,y)dx +Q(x,y)dy=0\]\[\text{要求:}\frac{\partial P}{\partial y}=\frac{\partial Q}{\partial x}\]
            \item 乘积分因子后可化全微分的方程\[M(x,y)dx +N(x,y)dy=0\]\[\mu M(x,y)dx +\mu N(x,y)dy=0\]\[\text{要求:}\frac{\partial (\mu M)}{\partial y}=\frac{\partial(\mu N)}{\partial x}\]

\end{enumerate}
    \end{enumerate}
    \item 二阶
    \begin{enumerate}
        \item 常系数线性方程
        \begin{enumerate}
            \item 齐次(特征根法求通解)\[y''+py'+qy=0\]\[\lambda ^2+p\lambda+q=0\]
            \begin{itemize}
                \item 相异一重实根\[\lambda _1 \neq \lambda _2\]
                \item 二重实根\[\lambda _1 = \lambda _2\]
                \item 一重共轭复根\[\lambda = \alpha \pm \beta i\]
            \end{itemize}
            \item 非齐次(可以待定系数的类型)(待定系数求特解)\[y''+py'+qy=f(x)\]
            \begin{enumerate}
                \item 多项式\[f(x)=P_n(x)\]
                \begin{itemize}
                    \item 0不是特征根
                    \item 0是单根
                    \item 0是重根
                \end{itemize}
                \item 指数函数\[f(x)=ae^{\alpha x}\]
                \begin{itemize}
                    \item \(\alpha\)不是特征根
                    \item \(\alpha\)是单根
                    \item \(\alpha\)是重根
                \end{itemize}
                \item 三角函数\[f(x)=a\cos \beta x +b\sin \beta x\]
                \begin{itemize}
                    \item \(\pm \beta i\)不是特征根
                    \item \(\pm \beta i\)是特征根
                \end{itemize}
                \item 多项式乘指数函数\[f(x)=P_n(x)e^{\alpha x}\]
                \begin{itemize}
                    \item \(\alpha\)不是特征根
                    \item \(\alpha\)是单根
                    \item \(\alpha\)是重根
                \end{itemize}
                \item 多项式乘指数函数乘三角函数\[f(x)=P_n(x)e^{\alpha x}(a\cos \beta x +b\sin \beta x)\]
                \begin{itemize}
                    \item \(\alpha \pm \beta i\)不是特征根
                    \item \(\alpha \pm \beta i\)是特征根
                \end{itemize}

            \end{enumerate}
            \item 非齐次(不可以待定系数求特解的类型)(常数变易法直接求通解)\[y''+py'+qy=f(x)\]
            \item 欧拉方程(仍然是$n$阶线性的,但不是常系数的)(可以化为常系数)\[a_0x^2y^{(2)}+a_1x^1y^{(1)}+a_3x^0y^{(0)}=f(x)\]
        \end{enumerate}
        \item 欧拉方程(仍然是$n$阶线性的,但不是常系数的)(可以化为常系数)\[a_0x^2y^{(2)}+a_1x^1y^{(1)}+a_3x^0y^{(0)}=f(x)\]
    
    \end{enumerate}
    \item 高阶
    \begin{enumerate}
        \item 常系数线性方程
        \begin{enumerate}
            \item 齐次(特征根法求通解)\[y^{(n)}+a_1y^{(n-1)}+a_2y^{(n-2)}+...+a_ny^{(0)}=0\]
            \begin{itemize}
                \item 一重实根
                \item $k$重实根$(k>1)$
                \item 一重共轭复根
                \item m重共轭复根$(m>1)$
            \end{itemize}
            \item 非齐次\[y^{(n)}+a_1y^{(n-1)}+a_2y^{(n-2)}+...+a_ny^{(0)}=f(x)\]
            \item 欧拉方程(仍然是$n$阶线性的,但不是常系数的)(可以化为常系数)\[a_0x^ny^{(n)}+a_1x^{n-1}y^{(n-1)}+...+a_nx^0y^{(0)}=f(x)\]
        \end{enumerate}
        \item 欧拉方程(仍然是$n$阶线性的,但不是常系数的)(可以化为常系数)\[a_0x^ny^{(n)}+a_1x^{n-1}y^{(n-1)}+...+a_nx^0y^{(0)}=f(x)\]
        \end{enumerate}
    \end{enumerate}
\subsection{一阶线性常系数常微分方程组}
\begin{enumerate}
    \item 齐次\[\vec v(t) = A \vec r(t)\]\[\text{其中}A\text{为常数系数构成的矩阵}\]
    \item 非齐次\[\vec v(t) = A \vec r(t)+\vec \beta (t)\]\[\text{其中}A\text{为常数系数构成的矩阵}\]

\end{enumerate}


\end{document}
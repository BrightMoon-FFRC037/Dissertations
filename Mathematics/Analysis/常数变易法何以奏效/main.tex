%\documentclass{article}
\documentclass[UTF8]{ctexart}
% Language setting
% Replace `english' with e.g. `spanish' to change the document language
\usepackage[english]{babel}

% Set page size and margins
% Replace `letterpaper' with `a4paper' for UK/EU standard size
\usepackage[a4paper,top=2cm,bottom=2cm,left=3cm,right=3cm,marginparwidth=1.75cm]{geometry}

% Useful packages
\usepackage{amsmath}
\usepackage{graphicx}
\usepackage[colorlinks=true, allcolors=blue]{hyperref}

\title{常数变易法何以奏效}
\author{Bright Moon}

\begin{document}
\maketitle



\section{二阶常系数线性微分方程的降阶}
\begin{equation}\label{eq:1}
    y''+py'+qy=f(x)
\end{equation}
\[\lambda ^2 +p\lambda+q=0\\\]
\[-p=\lambda_1+\lambda_2;\quad q=\lambda_1\lambda_2\]
代入公式\ref{eq:1}中,可以得到下式:
\[y''-\lambda _1y'-\lambda _2y'+\lambda _1 \lambda _2y=f(x)\]
\begin{equation}\label{eq:2}
(y''-\lambda _1y')-\lambda _2(y'-\lambda _1y)=f(x)
\end{equation}
令$z=y'-\lambda _1y$,代入公式\ref{eq:2}中得到:
\begin{equation}\label{eq:3}
z'-\lambda _2z=f(x)
\end{equation}
这一步实现了,二阶常系数线性微分方程,向一阶常系数线性微分方程的降阶。
\section{一阶常系数线性微分方程的常数变易}
令积分因子$\mu=e^{-\lambda _2x}$
\[\mu z'-\lambda _2 \mu z = \mu f(x)\]
\[z'e^{-\lambda _2x}-\lambda _2e^{-\lambda _2x}z=e^{-\lambda _2x}f(x)\]
\[(ze^{-\lambda _2x})'=e^{-\lambda _2x}f(x)\]
\[ze^{-\lambda _2x} = \int e^{-\lambda _2x}f(x)dx\]
\[z =e^{\lambda _2x} \int e^{-\lambda _2x}f(x)dx\]
\begin{equation}\label{eq:z}
    z =e^{\lambda _2x}\left ( \int _{x_0} ^x e^{-\lambda _2t}f(t)dt+C_1\right )
\end{equation}

其中后半部分就是常数$C_1$经过变易后的形式$C_1(x)$
\[C_1(x)=\left ( \int _{x_0} ^x e^{-\lambda _2t}f(t)dt+C_1\right )\]
当原方程是齐次方程的时候,$f(t)=0$,进而有:
\[C_1(x)=\left ( \int _{x_0} ^x e^{-\lambda _2t}f(t)dt+C_1\right )=\left ( \int _{x_0} ^x e^{-\lambda _2t}0dt+C_1\right )=0+ C_1=C_1\]
变易的常数退化为原来的形式(最后两步把新产生的常数合并到了原来的$C_1$中,仍记为$C_1$)。
\section{二阶常系数线性微分方程的常数变易}
\[y''+py'+qy=f(x)\]
等价于:
\[(y''-\lambda _1y')-\lambda _2(y'-\lambda _1y)=f(x)\]
等价于:
\[\begin{cases}
z'-\lambda _2z=f(x)\\
y'-\lambda _1y=z(x)\\
\end{cases}\]
用同样的,处理一阶的方法求解$y(x)$得到:
\[y = e^{\lambda _1 x}\int z e^{-\lambda _1 x}dx\]
\[y = e^{\lambda _1 x}\left (\int_{x_0}^{x} z e^{-\lambda _1 s}ds+C_2\right )\]
再代入求得的$z(x)$(公式\ref{eq:z})得到:

\[y = e^{\lambda _1 x}\left (\int_{x_0}^{x} e^{\lambda _2s}\left ( \int _{s_0} ^s e^{-\lambda _2t}f(t)dt+C_1\right ) e^{-\lambda _1 s}ds+C_2\right )\]
整理一下:
\[y = e^{\lambda _1 x}\left (\int_{x_0}^{x} \left ( \int _{s_0} ^s e^{-\lambda _2t}f(t)dt+C_1\right ) e^{\lambda _2s}e^{-\lambda _1 s}ds+C_2\right )\]
\begin{equation}\label{eq:key}
    y = e^{\lambda _1 x}\left (\int_{x_0}^{x} \left ( \int _{s_0} ^s e^{-\lambda _2t}f(t)dt+C_1\right ) e^{(\lambda _2-\lambda _1 )s}ds+C_2\right )
\end{equation}
\subsection{齐次情形$f(t)=0$}
\subsubsection{相异两根$\lambda _2 \neq \lambda _1$}
公式\ref{eq:key}可以化简为:
\[y = e^{\lambda _1 x}\left (\int_{x_0}^{x} \left ( \int _{s_0} ^s 0dt+C_1\right ) e^{(\lambda _2-\lambda _1 )s}ds+C_2\right )\]
\begin{equation}\label{eq:homo}
y = e^{\lambda _1 x}\left (\int_{x_0}^{x} C_1 e^{(\lambda _2-\lambda _1 )s}ds+C_2\right )
\end{equation}
把定积分多出来的常数,合并到$C_2$里构成新的$\tilde C_2$:
\[y = e^{\lambda _1 x}\left (\frac{C_1}{\lambda _2-\lambda _1}e^{(\lambda _2-\lambda _1 )x}+\tilde C_2\right )\]

把$\lambda _1,\ \lambda _2,\ C_1$合并成为新的常数,$\tilde C_1$:
\[y=\tilde C_1e^{\lambda _2x}+\tilde C_2e^{\lambda _1 x}\]
这种思路比先猜到两个解,然后再验证线性无关性(Wronski行列式非零),或者验证两个常数的独立性,要自然顺畅很多。
\subsubsection{相同两根$\lambda _2 =\lambda _1$}
这个时候公式\ref{eq:homo}可以化简为:
\[y = e^{\lambda _1 x}\left (\int_{x_0}^{x} C_1 e^{0}ds+C_2\right )\]
\[y = e^{\lambda _1 x}\left (\int_{x_0}^{x} C_1 ds+C_2\right )\]
把定积分多出来的常数,合并到$C_2$里构成新的$\tilde C_2$:
\[y = e^{\lambda _1 x}\left (C_1x +\tilde C_2\right )\]
最终有:
\[y= C_1xe^{\lambda _1x}+\tilde C_2e^{\lambda _1 x}\]

这就解释了为什么“遇事不决要乘一个$x$”。让这种无厘头的操作合乎于推理。
\subsection{非齐次情形$f(t)\neq 0$}
对公式\ref{eq:key}进行化简整理,把涉及到常数的部分往外提,得到:
\[y=\left(\int^x_{x_0}\left(\int_{s_0}^s f(t)e^{-\lambda _2t}dt\right)e^{(\lambda _2-\lambda _1)s}ds\right)e^{\lambda _1x}+\left(C_1\int^x_{x_0}e^{(\lambda _2-\lambda _1)s}ds\right)e^{\lambda _1x}+C_2e^\lambda _1 x\]
其中后两项,就是公式\ref{eq:homo}中的形式,所以可以借用上面的结论,化简后两项。
\subsubsection{相异两根$\lambda _2 \neq \lambda _1$}
\[y=\left(\int^x_{x_0}\left(\int_{s_0}^s f(t)e^{-\lambda _2t}dt\right)e^{(\lambda _2-\lambda _1)s}ds\right)e^{\lambda _1x}+\tilde C_1e^{\lambda _2x}+\tilde C_2e^{\lambda _1 x}\]
其中第一项负责让两个常数$\tilde C_1,\ \tilde C_2$发生变易,成为$\tilde C_1(x),\ \tilde C_2(x)$
\subsubsection{相同两根$\lambda _2 = \lambda _1$}
\[y=\left(\int^x_{x_0}\left(\int_{s_0}^s f(t)e^{-\lambda _2t}dt\right)e^{0}ds\right)e^{\lambda _1x}+ C_1xe^{\lambda _1x}+\tilde C_2e^{\lambda _1 x}\]
\[y=\left(\int^x_{x_0}\left(\int_{s_0}^s f(t)e^{-\lambda _2t}dt\right)ds\right)e^{\lambda _1x}+ C_1xe^{\lambda _1x}+\tilde C_2e^{\lambda _1 x}\]
同样地,第一项负责让两个常数$\tilde C_1,\ \tilde C_2$发生变易,成为$ C_1(x),\ \tilde C_2(x)$。\textbf{这就解释了常数变易法,为什么会是一种可行的方法,顺着什么样的思路可以想到常数变易法。}解微分方程,一些看起来必须要猜的结论,其实用推理的方法也可以得到。
















\end{document}